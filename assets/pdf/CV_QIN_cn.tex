\documentclass[letterpaper]{article}
\usepackage{xeCJK}
\usepackage[breaklinks,urlbordercolor={1 1 1}]{hyperref}
\usepackage{geometry}
\usepackage{tabularx}
\usepackage[resetlabels]{multibib}
\newcites{A}{\textsc{论文}}

% Comment the following lines to use the default Computer Modern font
% instead of the Palatino font provided by the mathpazo package.
% Remove the 'osf' bit if you don't like the old style figures.
\usepackage[T1]{fontenc}
\usepackage[sc,osf]{mathpazo}

% Set your name here
\def\name{\textbf{秦雨潇}}

% Replace this with a link to your CV if you like, or set it empty
% (as in \def\footerlink{}) to remove the link in the footer:
\def\footerlink{}

% The following metadata will show up in the PDF properties
\hypersetup{
  colorlinks = true,
  urlcolor = black,
  pdfauthor = {\name},
  pdftitle = {\name: Curriculum Vitae},
  pdfsubject = {Curriculum Vitae},
  pdfpagemode = UseNone
}

\geometry{
  body={6.5in, 9in},
  left=1.0in,
  top=1.25in
}

% Customize page headers
\pagestyle{myheadings}
\markright{\name}
\thispagestyle{empty}

% Custom section fonts
\usepackage{sectsty}
\sectionfont{\rmfamily\mdseries\Large}
\subsectionfont{\rmfamily\mdseries\itshape\large}

% Other possible font commands include:
% \ttfamily for teletype,
% \sffamily for sans serif,
% \bfseries for bold,
% \scshape for small caps,
% \normalsize, \large, \Large, \LARGE sizes.

% Don't indent paragraphs.
\setlength\parindent{0em}

% Make lists without bullets
\renewenvironment{itemize}{
  \begin{list}{}{
    \setlength{\leftmargin}{1.5em}
  }
}{
  \end{list}
}

\begin{document}

% Place name at left
{\huge \name}

% Alternatively, print name centered and bold:
%\centerline{\huge \name}

\vspace{0.2in}

\begin{minipage}{0.8\linewidth}
	副教授 \\
	中国西安\ 西北工业大学\ 电子信息学院\\
  \href{mailto:hi@yuxiaoq.in}{\tt hi@yuxiaoq.in} \\ 
  \href{www.yuxiaoq.in}{www.yuxiaoq.in}
\end{minipage}

%---------------------------------------------------------------
%	RESEARCH INTERESTS
%---------------------------------------------------------------

\section*{\textsc{研究兴趣}}
- 遥感的信号与图像处理;\vspace{.5em}\\
- 合成孔径雷达干涉遥感的信号处理、算法与软件开发;\vspace{.5em}\\
- InSAR 的时序分析、地表监测与工程应用。

%---------------------------------------------------------------
%	EDUCATION
%---------------------------------------------------------------

\section*{\textsc{教育经历}}
\begin{tabular}{ll}
2014 - 2018 & 博士,美国\ 普渡大学\ 土木工程学院\ 遥感与测绘专业\vspace{.5em}\\
%------------------------------------------------
2015 - 2017 & 硕士,美国\ 普渡大学\ 电子信息学院 通讯、数字信号与图像处理专业\vspace{.5em}\\
%------------------------------------------------
2010 - 2011 & 硕士,香港中文大学\ 太空与地球信息科学研究所\  地球系统科学专业 \vspace{.5em}\\
%------------------------------------------------
2006 - 2010 & 学士,北京大学\ 地球与空间科学学院\ 地理信息系统与遥感专业
\end{tabular}

%---------------------------------------------------------------
%	WORK & RESEARCH EXPERIENCE
%---------------------------------------------------------------

\section*{\textsc{科研工作经历}}
\begin{tabular}{ll}
2020 -      & 特聘副教授,中国西安\ 西北工业大学\ 电子信息学院  \vspace{.5em}\\
2018 - 2020 & 首席工程师, 荷兰代尔夫特\ SkyGeo Netherlands B.V. \vspace{.5em}\\
%------------------------------------------------
2014 - 2018 & 助理研究员,美国\ 普渡大学\ 土木工程学院 \vspace{.5em}\\
2011 - 2013 & 助理研究员,香港中文大学\ 太空与地球信息科学研究所
\end{tabular}

%---------------------------------------------------------------
%	SOCIAL ACTIVITIES
%---------------------------------------------------------------
\section*{\textsc{社会经历}}
\begin{tabular}{ll}
2007 - 2010     & 主席、理事、理事长, 北京大学自行车协会\vspace{.5em}\\
2011 - 2013     & 理事, 北京大学香港校友会\\
\end{tabular}

%---------------------------------------------------------------
%	PUBLICATIONS
%---------------------------------------------------------------

%\renewcommand{\refname}{\textsc{Publications}}
\makeatletter \renewcommand\@biblabel[1]{[#1]} \makeatother

\nociteA{QIN2020}
\nociteA{QIN2018}
\nociteA{QIN2018-2}
\nociteA{QIN2015}
\nociteA{QIN2013}
\nociteA{QIN2019}

\bibliographystyleA{ieeetr}
\bibliographyA{QIN}

% Footer
\begin{center}
  \begin{footnotesize}
    Last updated: \today \\
%    \href{\footerlink}{\texttt{\footerlink}}
  \end{footnotesize}
\end{center}

\end{document}
